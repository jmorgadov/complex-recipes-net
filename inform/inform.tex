%===================================================================================
% PREÁMBULO
%-----------------------------------------------------------------------------------
\documentclass[a4paper]{article}

%===================================================================================
% Paquetes
%-----------------------------------------------------------------------------------
\usepackage{amsmath}
\usepackage{amsfonts}
\usepackage{amssymb}
\usepackage{inform}
\usepackage[utf8]{inputenc}
\usepackage{listings}
\usepackage[pdftex]{hyperref}
\usepackage{caption}
\usepackage{subcaption}
%-----------------------------------------------------------------------------------
% Configuración
%-----------------------------------------------------------------------------------
\hypersetup{colorlinks,%
	    citecolor=black,%
	    filecolor=black,%
	    linkcolor=black,%
	    urlcolor=blue}

%===================================================================================



%===================================================================================
% Presentacion
%-----------------------------------------------------------------------------------
% Título
%-----------------------------------------------------------------------------------
\title{Análisis de redes complejas en el mundo de la cocina: una exploración de
las relaciones entre ingredientes y recetas}

%-----------------------------------------------------------------------------------
% Autores
%-----------------------------------------------------------------------------------
\author{\\
\name Jorge Morgado \email \href{mailto:jorge.morgadov@gmail.com}{jorge.morgadov@gmail.com}
	\AND
\name Roberto García \email \href{mailto:robegr42@gmail.com}{robegr42@gmail.com}
}

%-----------------------------------------------------------------------------------
% Tutores
%-----------------------------------------------------------------------------------
% \tutors{\\
% Dr. Tutor Uno, \emph{Centro} \\
% Lic. Tutor Dos, \emph{Centro}}

%-----------------------------------------------------------------------------------
% Headings
%-----------------------------------------------------------------------------------
\jcematcomheading{\the\year}{1-\pageref{end}}{A. Uno, A. Dos}

%-----------------------------------------------------------------------------------
\ShortHeadings{Análisis de redes complejas}{J. Morgado, R. García}
%===================================================================================



%===================================================================================
% DOCUMENTO
%-----------------------------------------------------------------------------------
\begin{document}

%-----------------------------------------------------------------------------------
% NO BORRAR ESTA LINEA!
%-----------------------------------------------------------------------------------
% \twocolumn[
%-----------------------------------------------------------------------------------

\maketitle

%===================================================================================
% Resumen y Abstract
%-----------------------------------------------------------------------------------
\selectlanguage{spanish} % Para producir el documento en Español

%-----------------------------------------------------------------------------------
% Resumen en Español
%-----------------------------------------------------------------------------------
\begin{abstract}

	El análisis de redes complejas es una técnica matemática que se utiliza para
	estudiar sistemas complejos que pueden ser representados mediante nodos y
	conexiones. Esta técnica se aplica en una amplia variedad de campos, como la
	biología, la física, la sociología, la informática y muchos otros. En el
	ámbito de las recetas de cocina, el análisis de redes complejas se utiliza
	para analizar las relaciones entre los ingredientes y las recetas. Estos
	estudios han permitido identificar patrones, tendencias en la forma en que
	se utilizan los ingredientes,  las combinaciones de sabores más populares y
	las recetas más comunes, lo que ha permitido a los chefs y amantes de la
	comida entender mejor las preferencias de los consumidores y desarrollar
	nuevas combinaciones de sabores y recetas.

\end{abstract}

%-----------------------------------------------------------------------------------
% English Abstract
%-----------------------------------------------------------------------------------
\vspace{0.5cm}

\begin{enabstract}

	Complex network analysis is a mathematical technique used to study complex
	systems that can be represented by nodes and connections. This technique is
	applied in a wide variety of fields such as biology, physics, sociology,
	computer science, and many others. In the field of cooking recipes, complex
	network analysis is used to analyze the relationships between ingredients
	and recipes. These studies have allowed the identification of patterns and
	trends in the way ingredients are used, the most popular flavor
	combinations, and the most common recipes. This has enabled chefs and food
	enthusiasts to better understand consumer preferences and develop new flavor
	combinations and recipes.

\end{enabstract}

%-----------------------------------------------------------------------------------
% Palabras clave
%-----------------------------------------------------------------------------------
\begin{keywords}
	Red compleja, grafo, ingredientes
\end{keywords}

%-----------------------------------------------------------------------------------
% Temas
%-----------------------------------------------------------------------------------
\begin{topics}
	Análisis de redes complejas
\end{topics}


%-----------------------------------------------------------------------------------
% NO BORRAR ESTAS LINEAS!
%-----------------------------------------------------------------------------------
\vspace{0.8cm}
% ]
%-----------------------------------------------------------------------------------


%===================================================================================

%===================================================================================
% Introducción
%-----------------------------------------------------------------------------------
\section{Introducción}\label{sec:intro}
%-----------------------------------------------------------------------------------

En los últimos años, el análisis de redes complejas se ha convertido en una
técnica matemática clave para entender la estructura y dinámica de sistemas
complejos que pueden ser representados por nodos y conexiones.

En particular, en el ámbito de las recetas de cocina, el análisis de redes
complejas se ha utilizado para analizar las relaciones entre los ingredientes y
las recetas. Esta técnica ha permitido identificar patrones y tendencias en la
forma en que se utilizan los ingredientes y las combinaciones de sabores más
populares. Además, ha permitido entender mejor la historia y la evolución de las
recetas, al analizar cómo los ingredientes y las combinaciones de sabores han
cambiado con el tiempo y cómo se han difundido a través de diferentes culturas y
regiones.

En una red compleja de recetas de cocina, los nodos representan los ingredientes
y las conexiones representan las relaciones entre ellos. Estas relaciones pueden
ser de diferentes tipos, como la frecuencia con la que se utilizan juntos los
ingredientes en diferentes recetas o la compatibilidad de los sabores.

En este trabajo, se explorará el análisis de redes complejas en el ámbito de las
recetas de cocina. Se describirán los fundamentos teóricos de la técnica y se
presentarán ejemplos de su aplicación en el estudio de las relaciones entre
ingredientes y recetas. También se explorarán las posibles aplicaciones futuras
de esta técnica en el desarrollo de nuevas combinaciones de sabores y recetas.
En conjunto, el análisis de redes complejas se presenta como una herramienta
valiosa para entender y explorar la complejidad de las relaciones entre los
ingredientes y las recetas en el mundo de la cocina.

%===================================================================================



%===================================================================================
% Desarrollo
%-----------------------------------------------------------------------------------
\section{Desarrollo}\label{sec:dev}
%-----------------------------------------------------------------------------------
  En esta sección (o secciones) incluya el contenido fundamental del artículo.
  No es necesario tener una sección nombrada \emph{Desarrollo}, por el contrario,
  nombre las secciones según el contenido que tratan.

%-----------------------------------------------------------------------------------
	\subsection{Organización del Documento}\label{sub:results}
%-----------------------------------------------------------------------------------
		Puede agregar secciones y subsecciones según sea necesario para organizar
		de manera más coherente su artículo. Tenga en cuenta que un documento más
		plano es más fácil de navegar y entender, pero las subsecciones relacionadas
		deberían estar agrupadas en una sección común.

		Los nombres de las secciones deben ir en mayúsculas, excepto para las
		preposiciones, conjunciones, y otros vocablos auxiliares.

		Empiece un nuevo párrafo cada vez que vaya a comenzar una idea nueva.

%-----------------------------------------------------------------------------------
	\subsection{Listas y Descripciones}\label{sub:lists}
%-----------------------------------------------------------------------------------
		Para producir listas enumeradas, utilice el siguiente estilo:
		\begin{enumerate}
			\item Primer Elemento
			\item Segundo Elemento
			%
			\begin {enumerate}
				\item {Segundo Elemento - Subítem Uno}
				\item {Segundo Elemento - Subítem Dos}
			\end {enumerate}
			%
		\end{enumerate}

%-----------------------------------------------------------------------------------
		Para producir descripciones, use el siguiente estilo:

%-----------------------------------------------------------------------------------
		\begin{description}
			\item [Primer Elemento] con su respectiva descripción.
			\item [Segundo Elemento] también con su respectiva descripción.
		\end{description}

%-----------------------------------------------------------------------------------
	\subsection{Figuras}\label{sub:figures}
%-----------------------------------------------------------------------------------
		Para producir cuerpos flotantes (figuras o tablas), asegúrese de numerar
		y etiquetar correctamente cada figura. Las referencias a las figuras deben
		estar correctamente etiquetadas. Por ejemplo, véase la Fig. \ref{fig:ex}\ldots

		\begin{figure}[h!]%
		\begin{center}
			\begin{tabular}{|c|c|c|} \hline
			 			& Método 1 	& Método 2 	\\ \hline
			A 			&  			&  			\\ \hline
			B			& 			& 			\\ \hline
			C 			& 			&  			\\ \hline
			\end{tabular}
		\caption{Figura de ejemplo. Recuerde especificar el origen de los datos que se muestran. \label{fig:ex}}
		\end{center}
		\end{figure}

%-----------------------------------------------------------------------------------
	\subsection{Código Fuente}\label{sub:listings}
%-----------------------------------------------------------------------------------
		Para producir código fuente, envuélvalo en una figura flotante y
		etiquételo correctamente. Por ejemplo, en la Fig. \ref{fig:code}
		se muestra un código bastante conocido\ldots

		% Configuración de Listings
		\lstset{keywordstyle=\color{blue}, basicstyle=\small}

		\begin{figure}[htb]%
			\begin{lstlisting}[language=c]%

    int main(int argc, char** argv)
    {
        // Imprimiendo "Hola Mundo".
        printf("Hello, World");
    }

			\end{lstlisting}
		\caption{Código fuente de ejemplo.\label{fig:code}}
		\end{figure}

%-----------------------------------------------------------------------------------
	\subsection{Referencias}
%-----------------------------------------------------------------------------------
  	Las referencias deben estar agrupadas en una sección al final del artículo,
  	y las citas numeradas correctamente, por ejemplo \cite{knuth} o \cite{goedel}.
  	Incluya toda la información importante de cada referencia, incluídos autor,
  	título, y notas de la edición. En caso de citar sitios web, además
  	de la URL, incluya la fecha en que fue consultado, como en \cite{wiki}. Numere 
  	las referencias según el orden en que se les cita.

%===================================================================================



%===================================================================================
% Conclusiones
%-----------------------------------------------------------------------------------
\section{Conclusiones}\label{sec:conc}

El análisis de redes complejas es una técnica matemática poderosa y versátil que
se aplica en una amplia variedad de campos, incluyendo el de las recetas de
cocina. En este trabajo, se muestra cómo esta técnica se utiliza para entender
las relaciones entre los ingredientes y las recetas, y cómo ha permitido
identificar patrones y tendencias en la forma en que se utilizan los
ingredientes y las combinaciones de sabores. Además, se ha explorado cómo estás
técnicas pueden utilizarse para desarrollar nuevas combinaciones de sabores y
recetas.

En resumen, el análisis de redes complejas es una técnica de análisis esencial
para entender mejor y explorar la complejidad de las relaciones entre los
ingredientes y las recetas en el mundo de la cocina.

%===================================================================================



%===================================================================================
% Recomendaciones
%-----------------------------------------------------------------------------------
\section{Recomendaciones}\label{sec:rec}

A continuación, se presentan algunas recomendaciones a tener en cuenta al
utilizar el análisis de redes complejas en el ámbito de las recetas de cocina:
\begin{itemize}
	\item Utilizar una metodología rigurosa y estandarizada para la construcción
	de la red compleja de ingredientes y recetas.
	\item Asegurarse de que la interpretación de los resultados sea coherente
	con el contexto cultural y culinario de las recetas analizadas.
	\item Agregar más información a las redes(grafos) analizados así como la
	exploración de otras variantes.
\end{itemize}

%===================================================================================



%===================================================================================
% Bibliografía
%-----------------------------------------------------------------------------------
\begin{thebibliography}{99}
%-----------------------------------------------------------------------------------
	\bibitem{knuth} Donald E. Knuth. \emph{The Art of Computer Programming}.
		Volume 1: Fundamental Algorithms (3rd~edition), 1997.
		Addison-Wesley Professional.

	\bibitem{goedel} Kurt Göedel. \emph{Über formal unentscheidbare Sätze der
		Principia Mathematica und verwandter Systeme, I}.
		Monatshefte für Mathematik und Physik 38.

	\bibitem{wiki} Wikipedia. URL: \href{http://en.wikipedia.org}
	  {http://en.wikipedia.org}.
		Consultado en \today.

%-----------------------------------------------------------------------------------
\end{thebibliography}

%-----------------------------------------------------------------------------------

\label{end}

\end{document}

%===================================================================================
